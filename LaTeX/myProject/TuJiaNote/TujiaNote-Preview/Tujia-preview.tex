%!TEX program = xelatex

%%%%%%%%%%%%%%%%%%%%%%%%%%%%
% founder 选项: 方正字体     %
% 原模板: ElegantLatex-Note %
% TexLive 2020 + XeLaTeX   %
%%%%%%%%%%%%%%%%%%%%%%%%%%%%

\documentclass[cn,blue,12pt,normal,founder]{elegantnote}

\title{土家族医药学复习笔记}
\author{shujuecn}
% \institute{https://github.com/shujuecn}
% \version{2.30}
\date{\zhtoday}

\usepackage{float}            % 取消浮动
\usepackage{array}            % 表格
\usepackage{makecell}         % 合并表格
% \usepackage{ctex}
\usepackage{ulem}             % 删除线 下划线
% \usepackage{soul}
% \usepackage{wrapfig}
% \usepackage{graphicx}

% 彩色预览版
\newcommand{\redt}[1]{\textcolor{red}{{}#1}}      % 自定义红色字体命令

\begin{document}

\maketitle

\section{绪论}

\subsection{土家族医学的特点}

\begin{enumerate}
  \item 以古朴哲学思想为指导,并受多元文化的影响;
  \item 以“三元学说”为核心构建土家族医药独特的理论体系;
  \item 口耳相传;
  \item 医药护一体化;
  \item “五术一体”的传统外治法;
  \item 药材多鲜用,擅蛇伤、骨伤科。
\end{enumerate}

\section{人体结构和功能}

\subsection{三元脏器的组成及功能}

\begin{enumerate}
  \item 上元脏器(头元,居上天,统摄人体气、血、神志,为\redt{三元之首}。)
  \begin{itemize}
    \item 脑:主神
    \item 心:主血
    \item 肺:主呼吸
  \end{itemize}
  \item 中元脏器(腹元,居腹内,水谷出入、水精和谷精化生之处,为\redt{人体供养之本}。)
  \begin{itemize}
    \item 肚:主受纳、消磨
    \item 肠:主饮食物的变化
    \item 肝:主水谷精微的生成
  \end{itemize}
  \item 下元脏器(足元,居下元,排泌尿液,孕精生成之所,为\redt{生命发生之根}。)
  \begin{itemize}
    \item 腰子:主孕精的生成
    \item 精脬:接受、排出孕精
    \item 养儿肠:接受孕精,与男子孕精结合而生胎,或化生月经
    \item 尿脬:主尿
  \end{itemize}
\end{enumerate}

\subsection{十窍的组成及功能}

\begin{enumerate}
  \item 眼:二窍,司视万物,辨五色。
  \item 鼻:二窍,司嗅味,进出气。
  \item 耳:二窍,司听声音。
  \item 口:一窍,司言语,纳吐。
  \item 肛门:一窍,司排便。
  \item 尿孔:一窍,司排尿。
  \item 汗孔:一小窍,司排泄汗液。
\end{enumerate}

\subsection{肢节的组成及功能}

\begin{enumerate}
  \item 肢体
  \begin{itemize}
    \item 组成:手肢(上肢)、脚肢(下肢)、腰肢(躯干)。
    \item 功能:维持人体的正常姿势。
  \end{itemize}
  \item 骨头
  \begin{itemize}
    \item 组成:长骨、大骨、短小骨。
    \item 功能:支撑人体、保护内脏、进行运动。
  \end{itemize}
  \item 榫(骨节、关节)
  \begin{itemize}
    \item 定义:由两块及以上骨头连接起来,使其保持活动机能的\redt{联合处}。
    \item 组成:上肢骨节、下肢骨节、腰肢骨节。
    \item 功能:联结骨头、维持肢体骨节的运动。
  \end{itemize}
\end{enumerate}

\subsection{筋的组成及功能}

\begin{enumerate}
  \item 组成:肉筋索(肌腱)、肉皮筋(筋膜)、麻筋(神经)。
  \item 功能:\redt{前两者}可约束骨节、主持运动、保护内脏。
\end{enumerate}

\subsection{血脉的组成及功能}

\begin{enumerate}
  \item 青筋(静脉):将消耗了部分谷气和清气后的青血\redt{向心}输送。
  \item 索筋(动脉):由心\redt{向全身}输送具有营养作用的血液。
\end{enumerate}

\subsection{经脉的组成及功能}

\begin{enumerate}
  \item 组成
  \begin{itemize}
    \item 阳脉、阴脉、手脉、足脉、胞脉、裤腰带脉。
  \end{itemize}
  \item 功能
  \begin{itemize}
    \item 沟通三元;
    \item 联系人体上下、内外、表里各组织器官;
    \item 感应传导信息、调节机体平衡。
  \end{itemize}
\end{enumerate}

\subsection{土家医对气的认识}

\begin{enumerate}
  \item 气是\redt{构成人体}的基本物质
  \begin{itemize}
    \item 清气、谷气和精、血共同运行于筋脉中,循行至全身,充养人体各组织器官。
  \end{itemize}
  \item 气是\redt{维持生命活动}的主要物质基础
  \begin{itemize}
    \item 气具有激发、推动的生理作用,是人体生理活动的动力所在。
    \item 气也是人体维持恒定体温的关键所在。
  \end{itemize}
  \item 气是\redt{产生疾病}的原因之一
  \begin{itemize}
    \item 各组织器官产生的废气不能排出,蓄成病气。
    \item 自然界的病气、瘟气侵犯人体而致病。
  \end{itemize}
\end{enumerate}

\subsection{土家医对血的认识}

\begin{enumerate}
  \item 血的化生
  \begin{itemize}
    \item 水精与谷精在\redt{上元心气}的作用下,化生成为血液。
  \end{itemize}
  \item 血的功能
  \begin{itemize}
    \item 通过筋脉的输布到达全身各处,起\redt{濡养}作用;
    \item 与谷气相合,输布于周身,起\redt{营养}作用;
    \item 与清气结合,刺激机体生命活动,起\redt{动力}作用;
  \end{itemize}
  \begin{note}
    血还与谷精共同参与\redt{孕精}的生成。
  \end{note}
  \item 血的分类
  \begin{itemize}
    \item 红血:具有濡养作用的血液
    \item 青血:被消耗了部分谷气和清气后的血液
    \item 污血:废气多、含有瘟气的血液
    \item 黑血:失去了濡养作用的死血(瘀血)
  \end{itemize}
\end{enumerate}

\subsection{土家医对精的认识}

精是具有\redt{营养机体}和\redt{维持生育}功能的精微物质。

\begin{enumerate}
  \item 谷精
  \begin{itemize}
    \item 生成来源:饮食物。
    \item 生成条件:维持正常的饮食、肚肠肝的功能正常。
    \item 主要功能:\uline{营养作用、生成血液、化生孕精}。
  \end{itemize}
  \item 水精
  \begin{itemize}
    \item 生成来源:水饮流质食物。
    \item 生成条件:维持正常的饮食、肚肠肝的功能正常。
    \item 主要功能:\uline{濡养作用、生成血液、调节体温}。
    \item 代谢途径:汗液、尿液、呼吸、大便。
  \end{itemize}
  \item 孕精
  \begin{itemize}
    \item 生成来源:谷精与血,在腰子的作用下生成。
    \item 生成条件:谷精与血充盛、腰子功能健旺。
    \item 主要功能:\uline{维持男女生殖功能}。
  \end{itemize}
\end{enumerate}

\subsection{气、血、精的环流及相互关系}

\begin{enumerate}
  \item 精与气:气能生精、精能化气。
  \item 精与血:精能化血、血能生精。
  \item 气与血:
  \begin{itemize}
    \item 气对血:气能生血、气能行血。
    \item 血对气:血能载气、血能生气。
  \end{itemize}
\end{enumerate}
\begin{note}
  精、气、血三者均能相互影响。
\end{note}

\section{土家医病因、病理}

\subsection{土家医对疾病病因的认识}

\begin{enumerate}
  \item 瘟气

  因自然界的气候变化失常而产生的一系列致病因素。

  \begin{itemize}
    \item 风
    \subitem 特性:变动无常、轻清开泄。
    \subitem 分类:热风、冷风、水风、内风。

    \item 寒
    \subitem 特性:寒冷、凝滞。
    \subitem 分类:外寒、内寒。
    \item 暑
    \subitem 特性:火热兼暑湿、夏季多发。
    \item 湿
    \subitem 特性:黏滞、水湿,与风寒暑相兼为病。
    \subitem 分类:风湿、寒湿、暑湿。
    \item 火
    \subitem 特性:炎热。
    \subitem 分类:外火、火毒、三元内火。
  \end{itemize}

  \item 伤食

  \begin{itemize}
    \item 饮食不洁
    \subitem 食用不洁、变质、有毒的食物。
    \item 饮食失度
    \subitem 饮食过量:肚肠无力消化,食积发病。
    \subitem 饮食不足:营养不足,气血精生成减少。
    \item 饮食偏嗜
    \subitem 饮食过冷、过热、长期偏食。
  \end{itemize}

  \item 劳伤

  \begin{itemize}
    \item 外劳伤
    \subitem 跌打损伤、砸压、烧灼、冷冻。
    \item 内劳伤
    \subitem 房事无节制、劳神过度。
  \end{itemize}

  \item 情志

  \begin{itemize}
    \item 原因:突然、强烈、持久的情绪刺激。
    \item 表现:三元脏器功能紊乱、气血运行紊乱、伤脑损神。
  \end{itemize}

  \item 毒伤

  \begin{itemize}
    \item 天毒(瘟毒)
    \subitem \redt{动植物}腐败所滋生的秽浊之气,可传染,致病力强。
    \item 蔫毒
    \subitem 自然界中存在的\redt{有形毒物}。
    \subitem 虫毒、食毒、水毒、草毒。
    \item 玍毒
    \subitem \redt{机体代谢产物}不能正常排出,蓄积体内而发病。
    \subitem 气毒、血毒、尿毒、粪毒、脓毒、痰毒、胎毒、巴达毒。
    \item 无名之毒
  \end{itemize}

  \item 内虚

  \begin{itemize}
    \item 原因:先天不足、后天失养,气血精亏虚。
    \item 表现:机体脏器失养,功能失常,多种慢性病症。
  \end{itemize}
\end{enumerate}

\subsection{气血失调的病理变化}

\begin{enumerate}
  \item 气的病理
  \begin{itemize}
    \item 气亏:三元之气不足,脏腑组织功能减退,抗病能力差。
    \item 气阻:三元之气流通不畅或阻滞,导致机体功能障碍。
    \item 气逆:三元之气向上冲逆。
  \end{itemize}

  \begin{note}
  「中医学」气的失常:气虚、气机失调(气滞、气逆、\uline{\redt{气陷}、\redt{气闭}、\redt{气脱}})。
  \end{note}

  \item 血的病理
  \begin{itemize}
    \item 血亏:失血过多、化源不足,血液减少,组织器官失养。
    \item 血瘀:血液在筋脉中运行不畅,或溢出脉外、停滞体内。
    \item 出血:筋脉受损,血液溢出,经十窍排出体外。
    \item 血热:三元脏器内火热盛,热侵血脉。
    \item 血寒:寒气内舍血脉,凝滞气机,血行不畅。
  \end{itemize}

  \begin{note}
  「中医学」血的失常:血虚、血行失常(血寒、血热、血瘀、出血)。
  \end{note}
\end{enumerate}

\subsection{冷热失衡的病理变化}

\begin{enumerate}
  \item 冷的病理
  % \begin{itemize}
  %   \item 上元心肺气冷
  %   \item 中元肚肠气冷
  %   \item 下元腰子与养儿肠冷
  % \end{itemize}
  \begin{itemize}
    \item 上元心肺气冷、中元肚肠冷、下元腰子与养儿肠冷。
  \end{itemize}

  \item 热的病理
  % \begin{itemize}
  %   \item 心肺热证
  %   \item 肚肠热证
  %   \item 十窍热证
  % \end{itemize}
  \begin{itemize}
    \item 心肺热证、肚肠热证、十窍热证。
  \end{itemize}

  \item 冷热转化
  \begin{itemize}
    \item 由热转冷
    \subitem 热气消退,冷气增长;治疗不当;体内气血损耗过多。
    \item 由冷转热
    \subitem 冷气消退,热气增长;用药过热,助长了体内热气。
  \end{itemize}
  \item 冷热交错
  \begin{itemize}
    \item 内外:外热内冷、外冷内热。
    \item 上下:上热下冷、上冷下热。
    \item 先后:先热后冷、先冷后热。
  \end{itemize}
\end{enumerate}

\begin{note}
  「中医学」除上述气血、冷热的病理外,还包括表里、虚实、津液、脏腑、经络等各种病理变化,并据此归类为不同的辩证方法,如八纲辨证、病性辨证、病位辩证等。
\end{note}

\section{土家医五诊法}

% 土家医五诊法:看诊、问诊、听诊、脉诊、摸诊。

\subsection{看诊}

看神色、眼、舌、耳筋及耳、鼻、嘴、发、皮、指、背腹、二便、妇女病。

\begin{enumerate}
  \item 看耳的内容
  \begin{itemize}
    \item 小儿耳后筋脉
    \subitem 露出一个“丫”形:走\redt{狗}胎
    \subitem 筋上有一黑点:走\redt{兔}胎
    \subitem 筋脉有猴子样凸起:走\redt{猴}胎
    \item 妇人耳后筋脉
    \subitem 红色:火气重
    \subitem 青色:风气重
    \subitem 紫色:阴内有脏物
    \subitem 红暗色:妇科经带疾病
    \subitem 青紫色:妇女玉宫或喜道内有肿块
  \end{itemize}
  \item 看手的内容
  \begin{itemize}
    \item 看指壳(指甲)的颜色
    \subitem 手指应四季之疾:拇指为全年,食指为春季,中指为夏季,无名指为秋季,小指为冬季。
    \subitem 指壳颜色:青、白、黄主寒,赤主火,黑主风。
    \subitem 小儿指壳:乌黑色为走胎,红色主痨伤病,黄色主肿痛,白色主亏血。
    \subitem 妇女指壳:紫黑为白带多。
    \item 看中指的青筋(男左女右)
    \subitem 见于第\redt{一}指节间:病\redt{轻}
    \subitem 见于第\redt{二}指节间:病\redt{重}
    \subitem 见于第\redt{三}指节间:病\redt{危}
  \end{itemize}
\end{enumerate}

\subsection{问诊}

问饮食、二便、筋脉骨节、七窍、妇女病。

\subsection{听诊}

听说话、呼吸、呻唤、咳嗽、咯声与肠鸣音、骨擦音、女科音。

\subsection{脉诊}

\begin{enumerate}
  \item 土家医常用脉种
  \begin{itemize}
    \item 骨脉、命脉、芳脉、天脉、虎脉、肘关脉、踏地脉、鞋带脉、指缝脉、太阳脉(五阴六阳脉)、地支脉。
  \end{itemize}
  \item 土家医脉学特点
  \begin{itemize}
    \item 脉种繁多,脉象脉形简练;
    \item 循时号脉;
    \item 诊脉手法独特;
    \item 多脉合诊。
  \end{itemize}
\end{enumerate}

\subsection{摸诊}

摸骨断、冷热、疱疮、肚子。

\section{土家医治则治法}

\subsection{土家医“七法八则”的治疗原则}

\begin{enumerate}
  \item 八则
  \begin{itemize}
    \item 寒则热之、热则寒之、亏则补之、实则泻之
    \item 阻则通之、肿则消之、惊则镇之、湿则祛之
  \end{itemize}
  \item 七法
  \begin{itemize}
    \item 汗、\redt{泻}、\redt{赶}、止、补、温、清。
  \end{itemize}
  \begin{note}
  中医八法:汗、吐、\redt{下}、和、温、清、\redt{消}、补。
  \end{note}
\end{enumerate}

\subsection{土家医治毒法}

\begin{itemize}
  \item \redt{攻}毒法、\redt{败}毒法、\redt{赶}毒法、\redt{清}毒法、\redt{排}毒法
  \item \redt{拔}毒法、\redt{化}毒法、\redt{散}毒法、\redt{调}毒法、\redt{放}(提)毒法
\end{itemize}

\subsection{土家医传统外治法}

治疗\redt{小儿疾病}的外治法,有\uline{灯火疗法、提风疗法、蛋滚疗法、推抹疗法、翻背掐筋} \\ \uline{疗法}等。

治疗\redt{风湿痹痛}的外治法,有\uline{药浴疗法、熏蒸疗法、滚袋熨贴疗法、摸油锅疗法、赶} \\ \uline{油火疗法、麝火针疗法、赶酒火疗法、搓药法、拔火罐疗法}等。

治疗\redt{骨伤疾病}的外治法,有\uline{外敷疗法、药浴疗法、搓药法、封刀接骨法}等。



\begin{enumerate}
  \item 药棒疗法
  \begin{itemize}
    \item 方法:将药制成棒状或将木棒浸入药汁,捶击、敲打患处。
    \item 功效:药物的浸入作用、药棒敲打的舒筋活络作用。
    \item 适应证:半边瘫、麻木、肌肉筋膜疼痛、颈肩痛、腰腿疼。
  \end{itemize}
  \item 灯火疗法
  \begin{itemize}
    \item 方法:用灯芯蘸植物油,直接或间接点烧穴位。
    % \item 分类:直接灯火、印(阴)灯火、半灯火、隔纸灯火。
    \item 适应证:惊风症、小儿走胎、黄肿包、屙肚子、肚子痛、受凉、脑壳痛、风气麻木、扭伤、疱疮初期。
  \end{itemize}
  \begin{note}
  治疗疾病最广泛的外治法,治疗\redt{小儿病}必不可少的手段。
  \end{note}
  \item 赶酒火疗法
  \begin{itemize}
    \item 方法:用双手反复抓取点燃的药酒,在患处按揉。
    \item 功效:赶风除湿、舒筋活血。
    \item 适应证:风气病,如风湿麻木、冷骨风、骨节风、寒气内停、半边风。
    \item \redt{禁忌:}皮下有明显的血肿、高血压、心脏病患者忌用。
  \end{itemize}
  \item 提风疗法
  \begin{itemize}
    \item 方法:在熟鸡蛋中放入适量的药物,滴入桐油,紧敷于肚脐上。
    \item 功效:赶风、赶寒、赶热、赶气。
    \item 适应证:小儿因风寒风热引起的发热、抽筋或屙肚子、肚子胀、肚子痛及消化不良等症。
  \end{itemize}
  \item 蛋滚疗法
  \begin{itemize}
    \item 方法:将鸡蛋煮熟,在患者肚子上来回滚动。
    \item 功效:温里赶寒、赶食、吸毒。
    \item 适应证:风寒或停食所致肚子痛、肚子饱胀,饮食不洁所致呕吐、屙肚子等。
    \item \redt{禁忌:}虫积或火盛而致的大便硬结、肚子胀、肚子痛等。
  \end{itemize}
  \item 刮痧疗法
  \begin{itemize}
    \item 方法:用刮痧板蘸取植物油或清水,反复刮动摩擦穴位或体表肌肤。
    \item 功效:发散解表、舒筋活络、调整肚肠等功能。
    \item 适应证:中暑、伤风、伤寒发凉、发热、喉蛾等。
  \end{itemize}
  \item 封刀接骨法
  \begin{itemize}
    \item 复位基本手法:一揉摸,二捏位,三摇拐,四抵崴。
    \item 骨折固定法:鲜鸡接骨法、杉树皮接骨法、竹片接骨法、柳木接骨法、泡桐木接骨法、纸壳接骨法。
    \item 外敷药物:三百棒、独正岗、刺老苞、爬地麻、内红消、八月拿、黑虎七。
  \end{itemize}
\end{enumerate}

\section{土家医常用药}

% 土家药物学基本理论:

\begin{enumerate}
  \item 命名特点
  \begin{itemize}
    \item 形态、功能主治、味道、生活环境特征、药用部位、颜色特点、植物名。
  \end{itemize}
  \item 药物性能
  \begin{itemize}
    \item 三性:凉(寒)、温(热)、平。
    \item 八味:酸、甜、苦、辣、咸、涩、麻、淡。
  \end{itemize}
\end{enumerate}

% \begin{itemize}

% \end{itemize}

\subsection{发表药}

又称解表药,以\redt{赶风赶热}为主,辣散轻扬,行肌表,有促进肌体发汗,使表寒表热之邪由汗出而解的作用。兼有利尿消肿、止咳、止痛、消疮等作用。

\begin{table}[H]
  \begin{tabular}{|l|l|}
  \hline
  生姜 & 赶风、赶寒、赶气,解毒 \\ \hline
  野葛根 & 解肌透疹,止渴     \\ \hline
  \end{tabular}
\end{table}

\begin{note}
  「中药学」生姜:解表散寒,温中止呕,化痰止咳,解鱼蟹毒。
\end{note}

\subsection{赶风药}

又称风湿药,以驱除筋骨、骨肉之间的\redt{风寒湿邪},解除\redt{痹痛}为主,有驱风散寒除湿、舒筋活络、强筋健骨等作用。

\begin{table}[H]
  \begin{tabular}{|l|l|}
  \hline
  江边一碗水  & 驱风除湿,破瘀散结,止痛,解毒 \\ \hline
  天王七     & 驱风除湿,活血调经,止痛,止血     \\ \hline
  野乌头     & 驱风除湿,活血化瘀               \\ \hline
  \end{tabular}
\end{table}

\subsection{败火药}

又称败毒药,以消除人体\redt{火毒}为主,有泻火、燥湿、凉血、解毒、清虚火等功能。

\begin{table}[H]
  \begin{tabular}{|l|l|}
  \hline
  三颗针    & 败火燥湿,泻火解毒     \\ \hline
  八角莲    & 败火散结,散瘀止痛     \\ \hline
  小四块瓦  & 败火解毒,镇痉         \\ \hline
  海螺七*    & 败火解毒,止痛,解痉,\redt{活血化淤}    \\ \hline
  马桑根*    & 清热解毒,驱风除湿,镇痛,杀虫   \\ \hline
  \end{tabular}
\end{table}

\subsection{赶食药}

又称消食药,以消积导滞,促进消化,治疗饮食积滞为主,有开胃和中的作用。

\begin{table}[H]
  \begin{tabular}{|l|l|}
  \hline
  隔山消* & 健脾消食,解毒消痈,养阴补虚,利尿。 \\ \hline
  \end{tabular}
\end{table}

\subsection{泻下药}

能引起腹泻或润滑大肠,促进排便的药物,有泻下通便,排除胃物积滞、燥屎及有害物质,清热泻火,逐水消肿等作用。


\begin{table}[H]
  \begin{tabular}{|l|l|}
  \hline
  血丝大黄 & 泻下通便,败火解毒,活血散瘀  \\ \hline
  \end{tabular}
\end{table}

\begin{note}
「中药学」大黄:泻下攻积,清热泻火,凉血解毒,止血,逐瘀通经,利湿退黄。
\end{note}

\subsection{打伤药}

治疗各种暴力所致\redt{人体内外受伤}的药物,有活血化淤,消肿止痛,接筋续骨,止血生肌敛疮等作用。除跌打损伤外,也可用于其他一般血瘀病症。

\begin{table}[H]
  \begin{tabular}{|l|l|}
  \hline
  接骨丹* & 舒血活络,驱风除湿 \\ \hline
  八棱麻* & \redt{茎叶:}利尿消肿,活血止痛;\redt{根:}散瘀消肿,祛风活络 \\ \hline
  头顶一颗珠 & 镇惊安神,驱风活血止痛 \\ \hline
  \end{tabular}
\end{table}

\subsection{止痛药}

以调和气血,从而达到止痛作用的药物,多具\redt{辣味},主治\redt{气滞血瘀}所致的痛症,也可用于其他瘀血症。

\begin{table}[H]
  \begin{tabular}{|l|l|}
  \hline
  三百棒* & 活血舒筋,消肿止痛 \\ \hline
  四块瓦* & 舒筋活络,驱风除湿,消肿止痛 \\ \hline
  \makecell[l]{半截烂* \\ \redt{雪里见}} & 消肿解毒,驱风除湿 \\ \hline
  刺老苞 & 驱风除湿,散瘀消肿,活血止血,利尿止痛 \\ \hline
  \end{tabular}
\end{table}

\subsection{消水药}

又称利水药,以通利小便,渗出水液治疗\redt{水液内停}病症为主,具有利水消肿,利尿通淋,利湿退黄等作用。

\begin{table}[H]
  \begin{tabular}{|l|l|}
  \hline
  克马叶 & 利水消肿 \\ \hline
  \end{tabular}
\end{table}

\subsection{蛇药}

以治疗毒蛇咬伤为主的药物。

\begin{table}[H]
  \begin{tabular}{|l|l|}
  \hline
  半边莲 & 解蛇毒,利水消肿 \\ \hline
  蛇不过* & 败火解毒,赶风止咳,杀虫止痒,\redt{利湿消肿}  \\ \hline
  黄瓜香* & 败火解毒,排脓,\redt{消肿} \\ \hline
  \end{tabular}
\end{table}

\subsection{止咳药}

能制止或减轻咳吼,祛除痰涎或消化痰积的药物。

\begin{table}[H]
  \begin{tabular}{|l|l|}
  \hline
  三步跳 & \redt{内用:}燥湿化痰,降逆止咳,消痛散结;\redt{外用:}消肿止痛     \\ \hline
  \end{tabular}
\end{table}

\subsection{补虚药}

又称补养药,补益药,补药。主用于三元亏损,气、血虚弱的疾患。具有补养人体,\redt{扶正祛邪}的作用。

\begin{table}[H]
  \begin{tabular}{|l|l|}
  \hline
  罗汉七 & 补脾润肺,养阴生津 \\ \hline
  野党参* & 健中元,补肺气,祛痰止咳 \\  \hline
  恩施巴戟 & 补肾壮阳,强筋壮骨 \\ \hline
  \end{tabular}
\end{table}

\subsection{活血药}

通畅血行,消散淤血为主要作用的药物,善于走散通行,适用于一切\redt{瘀血阻滞}之证。

\begin{table}[H]
  \begin{tabular}{|l|l|}
  \hline
  冷水七 & 散瘀消肿,止血止痛,败火驱风 \\ \hline
  \makecell[l]{独正岗* \\ \redt{蛇葡萄}} & 活血散瘀,消肿止痛  \\  \hline
  \end{tabular}
\end{table}

\subsection{止血药}

能够制止人体内外\redt{各种出血}的药物,具有凉血止血、化瘀止血、收敛止血、温经止血等作用,适用于内外各种出血病证。

\begin{table}[H]
  \begin{tabular}{|l|l|}
  \hline
  文王一支笔 & 清热解毒,凉血止血,固肾涩精,\redt{解酒}    \\ \hline
  白三七* & 止血,散瘀通络,\redt{滋补强壮}   \\   \hline
  红子 & \redt{果:}健脾消积,活血止血;\redt{叶:}清热解毒;\redt{根:}清热凉血  \\  \hline
  \end{tabular}
\end{table}

\subsection{妇科药及喜药}

用于治疗妇科疾病及产科病证的药物,具有活血调经、安胎、止带或者堕胎的作用。

\begin{table}[H]
  \begin{tabular}{|l|l|}
  \hline
  扇子草 & 活血调经,散瘀止痛 \\ \hline
  \end{tabular}
\end{table}

\end{document}
